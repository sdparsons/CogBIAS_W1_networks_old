\documentclass[man,floatsintext]{apa6}
\usepackage{lmodern}
\usepackage{amssymb,amsmath}
\usepackage{ifxetex,ifluatex}
\usepackage{fixltx2e} % provides \textsubscript
\ifnum 0\ifxetex 1\fi\ifluatex 1\fi=0 % if pdftex
  \usepackage[T1]{fontenc}
  \usepackage[utf8]{inputenc}
\else % if luatex or xelatex
  \ifxetex
    \usepackage{mathspec}
  \else
    \usepackage{fontspec}
  \fi
  \defaultfontfeatures{Ligatures=TeX,Scale=MatchLowercase}
\fi
% use upquote if available, for straight quotes in verbatim environments
\IfFileExists{upquote.sty}{\usepackage{upquote}}{}
% use microtype if available
\IfFileExists{microtype.sty}{%
\usepackage{microtype}
\UseMicrotypeSet[protrusion]{basicmath} % disable protrusion for tt fonts
}{}
\usepackage{hyperref}
\hypersetup{unicode=true,
            pdftitle={Cognitive correlates of mental health in adolescence: A network analysis approach},
            pdfauthor={Sam Parsons, Annabel Songco, Charlotte Booth, \& Elaine Fox},
            pdfkeywords={keywords},
            pdfborder={0 0 0},
            breaklinks=true}
\urlstyle{same}  % don't use monospace font for urls
\usepackage{graphicx,grffile}
\makeatletter
\def\maxwidth{\ifdim\Gin@nat@width>\linewidth\linewidth\else\Gin@nat@width\fi}
\def\maxheight{\ifdim\Gin@nat@height>\textheight\textheight\else\Gin@nat@height\fi}
\makeatother
% Scale images if necessary, so that they will not overflow the page
% margins by default, and it is still possible to overwrite the defaults
% using explicit options in \includegraphics[width, height, ...]{}
\setkeys{Gin}{width=\maxwidth,height=\maxheight,keepaspectratio}
\IfFileExists{parskip.sty}{%
\usepackage{parskip}
}{% else
\setlength{\parindent}{0pt}
\setlength{\parskip}{6pt plus 2pt minus 1pt}
}
\setlength{\emergencystretch}{3em}  % prevent overfull lines
\providecommand{\tightlist}{%
  \setlength{\itemsep}{0pt}\setlength{\parskip}{0pt}}
\setcounter{secnumdepth}{0}
% Redefines (sub)paragraphs to behave more like sections
\ifx\paragraph\undefined\else
\let\oldparagraph\paragraph
\renewcommand{\paragraph}[1]{\oldparagraph{#1}\mbox{}}
\fi
\ifx\subparagraph\undefined\else
\let\oldsubparagraph\subparagraph
\renewcommand{\subparagraph}[1]{\oldsubparagraph{#1}\mbox{}}
\fi

%%% Use protect on footnotes to avoid problems with footnotes in titles
\let\rmarkdownfootnote\footnote%
\def\footnote{\protect\rmarkdownfootnote}


  \title{Cognitive correlates of mental health in adolescence: A network analysis approach}
    \author{Sam Parsons\textsuperscript{1}, Annabel Songco\textsuperscript{1}, Charlotte Booth\textsuperscript{1}, \& Elaine Fox\textsuperscript{1}}
    \date{}
  
\shorttitle{Combined Cognitive Bias Hypothesis Network}
\affiliation{
\vspace{0.5cm}
\textsuperscript{1} University of Oxford}
\keywords{keywords\newline\indent Word count: X}
\usepackage{csquotes}
\usepackage{upgreek}
\captionsetup{font=singlespacing,justification=justified}

\usepackage{longtable}
\usepackage{lscape}
\usepackage{multirow}
\usepackage{tabularx}
\usepackage[flushleft]{threeparttable}
\usepackage{threeparttablex}

\newenvironment{lltable}{\begin{landscape}\begin{center}\begin{ThreePartTable}}{\end{ThreePartTable}\end{center}\end{landscape}}

\makeatletter
\newcommand\LastLTentrywidth{1em}
\newlength\longtablewidth
\setlength{\longtablewidth}{1in}
\newcommand{\getlongtablewidth}{\begingroup \ifcsname LT@\roman{LT@tables}\endcsname \global\longtablewidth=0pt \renewcommand{\LT@entry}[2]{\global\advance\longtablewidth by ##2\relax\gdef\LastLTentrywidth{##2}}\@nameuse{LT@\roman{LT@tables}} \fi \endgroup}


\usepackage{lineno}

\linenumbers
\usepackage{float}
\floatplacement{figure}{H}
\raggedbottom

\authornote{Add complete departmental affiliations for each author here. Each new line herein must be indented, like this line.

Enter author note here.

Correspondence concerning this article should be addressed to Sam Parsons, Postal address. E-mail: \href{mailto:sam.parsons@psy.ox.ac.uk}{\nolinkurl{sam.parsons@psy.ox.ac.uk}}}

\abstract{
This is my abstract


}

\begin{document}
\maketitle

\hypertarget{comparing-groups-high-and-low-in-positive-mental-health}{%
\subsection{Comparing groups high and low in positive mental health}\label{comparing-groups-high-and-low-in-positive-mental-health}}

Figure 1 presents regularised partial correlations amongst interpretation and memory biases.

\includegraphics{script_files/figure-latex/plotting_highlow-1.pdf}

\hypertarget{network-comparisons}{%
\subsubsection{network comparisons}\label{network-comparisons}}

We compared the estimated networks using NetworkComparisonTest with 1000 iterations. Global strength in the high MH group (0.37) differed from that in the low MH group (1.70), \emph{p} = .013. There was no significant difference between global strength in the low MH group and the mid MH group (0.73), \emph{p} = .345; nor between the mid MH and high MH groups, \emph{p} = .361.

\newpage

\hypertarget{including-mental-health-in-the-model}{%
\subsection{Including Mental Health in the model}\label{including-mental-health-in-the-model}}

We explored the difference in models between high and low groups using the mgm package. Figure 2 presents the network including mental health as a categorical variable (only for high and low MH groups).

Note. the shaded area of the \enquote{pie} is the predicability of that node, i.e.~the variance explained in that variable by the rest of the network. (I also need to include a more detailed explanation of why MH is different here as a categorical variable).

\includegraphics{script_files/figure-latex/mgm-1.pdf}

\hypertarget{we-then-set-mental-health-to-be-a-moderator-of-the-network}{%
\subsection{we then set mental health to be a moderator of the network}\label{we-then-set-mental-health-to-be-a-moderator-of-the-network}}

\includegraphics{script_files/figure-latex/unnamed-chunk-3-1.pdf}

The square nodes link to three nodes. First, they each link to MH as the moderating variable. The two other nodes linked to indicate the edge that is moderated by the MH variable, e.g.~the relationship between Positive and Negative memory biases.

\hypertarget{we-followed-this-by-using-mh-as-a-linear-moderator-and-included-the-full-sample}{%
\subsection{we followed this by using MH as a linear moderator, and included the full sample}\label{we-followed-this-by-using-mh-as-a-linear-moderator-and-included-the-full-sample}}

\includegraphics{script_files/figure-latex/unnamed-chunk-5-1.pdf}

this network shows largely the same pattern as splitting by the high and low group. Some of the edges appear to be moderated by mental health.

\hypertarget{this-was-followed-by-examining-the-stability-of-the-network}{%
\subsection{This was followed by examining the stability of the network}\label{this-was-followed-by-examining-the-stability-of-the-network}}

\includegraphics{script_files/figure-latex/centralityplot-1.pdf}

The centrality plot provides an indication of how important each variable is to the network.

\includegraphics{script_files/figure-latex/edge_stability-1.pdf}

This plot provides a visualisation of the bootstrapped edge strenghts of all edges

\includegraphics{script_files/figure-latex/unnamed-chunk-7-1.pdf}

I'm not 100\% what this next one is just yet

\includegraphics{script_files/figure-latex/edge_differences-1.pdf}

this provides an indication of all differences between edges.

\includegraphics{script_files/figure-latex/centrality_stability-1.pdf}

plots the stability of the centrality indices, and next are the actual indices

\newpage

\hypertarget{references}{%
\section{References}\label{references}}

\begingroup
\setlength{\parindent}{-0.5in}
\setlength{\leftskip}{0.5in}

\hypertarget{refs}{}

\endgroup


\end{document}
